\documentclass[a4paper,11pt]{article}

% Packages
\usepackage[utf8]{inputenc}
\usepackage[T1]{fontenc}
\usepackage{lmodern}
\usepackage[margin=2.5cm]{geometry}
\usepackage{graphicx}
\usepackage{xcolor}
\usepackage{listings}
\usepackage{hyperref}
\usepackage{enumitem}
\usepackage{booktabs}
\usepackage{caption}
\usepackage{float}
\usepackage{setspace}
\usepackage{titlesec}
\usepackage{fancyhdr}

\pagestyle{fancy}
\fancyhf{} % Clear all header/footer fields

% Define header style
\fancyhead[L]{205.2 - Beyond relational databases}
\fancyhead[C]{Lab 2 - Chirp}
\fancyhead[R]{Jeremy Duc}
\fancyfoot[C]{\thepage}
\renewcommand{\headrulewidth}{0.4pt}
\renewcommand{\footrulewidth}{0.4pt}

% Set page style for first pages (title, TOC)
\fancypagestyle{plain}{%
  \fancyhf{}%
  \fancyfoot[C]{\thepage}%
  \renewcommand{\headrulewidth}{0pt}%
}

% Better section formatting
\titleformat{\section}
{\normalfont\Large\bfseries}{\thesection}{1em}{}[\titlerule]

% Better paragraph spacing
\setlength{\parindent}{0pt}
\setlength{\parskip}{6pt plus 2pt minus 1pt}

% Better table spacing
\renewcommand{\arraystretch}{1.3}

% Colors
\definecolor{codegreen}{rgb}{0,0.6,0}
\definecolor{codegray}{rgb}{0.5,0.5,0.5}
\definecolor{codepurple}{rgb}{0.58,0,0.82}
\definecolor{backcolour}{rgb}{0.95,0.95,0.95}

% Code listing style
\lstdefinestyle{mystyle}{
    backgroundcolor=\color{backcolour},   
    commentstyle=\color{codegreen},
    keywordstyle=\color{magenta},
    numberstyle=\tiny\color{codegray},
    stringstyle=\color{codepurple},
    basicstyle=\ttfamily\footnotesize,
    breakatwhitespace=false,         
    breaklines=true,                 
    captionpos=b,                    
    keepspaces=true,                 
    numbers=left,                    
    numbersep=5pt,                  
    showspaces=false,                
    showstringspaces=false,
    showtabs=false,                  
    tabsize=2,
    frame=single,
    framesep=5pt,
    aboveskip=10pt,
    belowskip=10pt
}
\lstset{style=mystyle}

% Document metadata
\title{
    \vspace{-1.5cm}
    \Huge{\textbf{Report for Lab 2 : Chirp }} \\
    \Large{205.2 - Beyond relational databases}
}
\author{Jeremy Duc}
\date{30 March 2025}

\begin{document}

\maketitle

\tableofcontents
\clearpage
\pagestyle{fancy}
\newpage

\section{Introduction and Motivation}

\subsection{Context}
This report presents the design, implementation, and evaluation of Chirp (Compact Hub for Instant Real-time Posting), a simplified Twitter clone developed as part of the "205.2 Beyond relational databases" course. The project demonstrates the application of key-value database concepts using Redis, moving beyond traditional relational database paradigms to explore alternative data modeling approaches.

\subsection{Project Objectives}
The primary objectives of this laboratory project were:
\begin{enumerate}
    \item Learn to model a practical data-intensive application as a key-value database
    \item Translate application requirements into implementation tasks and data modeling
    \item Implement and interact with a key-value store from a programmatic perspective
\end{enumerate}

\subsection{Motivation}
Social media platforms represent one of the most challenging use cases for database systems due to their high write throughput, complex data relationships, and need for real-time access. By implementing a Twitter-like service with Redis, this project provides hands-on experience with non-relational database patterns that can handle these requirements effectively. 

Twitter (now X) serves as an excellent model for this exercise since it has well-defined core functionalities that lend themselves to key-value representation. The ability to post short messages, follow users, and retrieve timelines aligns well with the strengths of key-value stores like Redis, which excel at fast read/write operations and sorted collections.

\section{Project Repository}

\subsection{Repository Link}
The complete codebase for this project is available at:
\begin{center}
\url{https://github.com/jijiduc/chirp-redis-keyvalue-lab}
\end{center}

\subsection{Repository Structure}
The repository follows a clean, professional organization with the following structure:
\begin{verbatim}
chirp-redis-keyvalue-lab/
|-- README.md                 # Installation and usage instructions
|-- requirements.txt          # Main dependencies
|-- requirements-dev.txt      # Development dependencies
|-- src/                      # Main source code 
|   |-- app/                  # Application code
|   |   |-- __init__.py
|   |   |-- chirp_app.py      # Command-line application
|   |   `-- streamlit_app.py  # Web application
|   `-- models/               # Redis data models
|       |-- __init__.py      
|       `-- redis_model.py    # Core Redis data model implementation
|-- scripts/                  # Utility scripts
|   |-- import_data.py        # Data import script
|   |-- process_jsonl.py      # Data processing script
|   |-- reset_db.py           # Database reset script
|   |-- run_app.py            # Application launcher
|   |-- run_tests.py          # Test runner
|   `-- fix_engagement.py     # Script to add engagement metrics
|-- data/                     # Generated data
|   `-- processed/            # Processed data directory
`-- tests/                    # Test suite
    |-- __init__.py
    |-- conftest.py           # Pytest configuration
    |-- test_redis_model.py   # Tests for Redis model
    |-- test_import_data.py   # Tests for data import
    `-- test_streamlit_app.py # Tests for web interface
\end{verbatim}

\subsection{Installation and Usage}
To install and run the application:

\begin{lstlisting}[language=bash, caption=Installation and setup commands]
# Clone the repository
git clone https://github.com/jijiduc/chirp-redis-keyvalue-lab.git
cd chirp-redis-keyvalue-lab

# Install required dependencies
pip install -r requirements.txt

# Optional: Install development dependencies for testing
pip install -r requirements-dev.txt

# Start Redis (if not running)
sudo systemctl start redis-server  # On Linux
# Or manually: redis-server

# Process and import sample data
python scripts/process_jsonl.py --generate-sample
python scripts/import_data.py ./data/processed/sample_english_tweets.json --add-engagement

# Run the command line application
python scripts/run_app.py

# Run the web interface
streamlit run src/app/streamlit_app.py
\end{lstlisting}

The README.md file in the repository contains comprehensive instructions for setting up the development environment, running the application, and executing the test suite. It also provides detailed information about the available commands and how to interact with the application.

\newpage
\section{Requirements Clarification and Assumptions}

\subsection{Core Requirements}
Based on the laboratory instructions, the minimal requirements for the Chirp application were:

\begin{itemize}
    \item \textbf{Following/followers}: Each user can have followers and follow other users
    \item \textbf{Chirps}: Users can post small text-only messages in English
    \item \textbf{Rankings}: The system must track and display various rankings:
    \begin{itemize}
        \item Top 5 users with highest follower counts
        \item Top 5 users with most chirps
        \item List of 5 latest chirps
    \end{itemize}
\end{itemize}

\subsection{Assumptions}
Several assumptions were made to guide the implementation:

\begin{enumerate}
    \item \textbf{Unidirectional relationship model}: Following is unidirectional, meaning if user A follows user B, it doesn't imply that B follows A
    \item \textbf{Limited timeline size}: To prevent memory exhaustion, the timeline will be capped at 100,000 entries
    \item \textbf{No tweet deletion}: For simplicity, the initial version doesn't support chirp deletion
    \item \textbf{Simple authentication}: User authentication is not implemented in this version
    \item \textbf{English-only content}: As specified in the requirements, only English chirps are considered
    \item \textbf{Engagement metrics}: Additional engagement metrics (likes, rechirps) were added to make the application more realistic
\end{enumerate}

\subsection{Design Decisions}
Several key decisions guided the implementation strategy:

\begin{enumerate}
    \item \textbf{Python as primary language}: Python was chosen for its excellent Redis library support and ability to rapidly prototype the application
    \item \textbf{Separation of concerns}: The implementation strictly separates the data model, command-line interface, and web interface to ensure maintainability
    \item \textbf{Streamlit for web interface}: Streamlit was selected for the web interface due to its simplicity and rapid development capabilities
    \item \textbf{Redis as the sole database}: All data is stored in Redis with no secondary storage systems
    \item \textbf{Unit testing}: Comprehensive unit tests were implemented to ensure reliability
    \item \textbf{Import pipeline}: A data import pipeline was created to populate the system with realistic Twitter data
\end{enumerate}

\newpage
\section{Data Modeling as Key-Value}

\subsection{Key Design Principles}
The data model was designed around several key principles specific to key-value databases:

\begin{enumerate}
    \item \textbf{Denormalization}: Data is intentionally duplicated where necessary to optimize read performance. For example:
    \begin{itemize}
        \item Username is stored in both user records and chirp records to avoid additional lookups
        \item Follower and chirp counts are stored in user records instead of being calculated on demand
        \item Engagement metrics (likes, retweets) are stored directly in chirp records
    \end{itemize}
    
    \item \textbf{Composite keys}: Meaningful prefixes are used to organize related data:
    \begin{itemize}
        \item \texttt{users:\{user\_id\}} for user profiles
        \item \texttt{chirp:\{chirp\_id\}} for individual posts
        \item \texttt{users:top\_followers} for follower rankings
        \item \texttt{chirps:timeline} for the global timeline
    \end{itemize}
    
    \item \textbf{Appropriate data structures}: Redis data types are selected based on access patterns:
    \begin{itemize}
        \item Hashes for complex entities (users, chirps)
        \item Sorted sets for rankings and timeline (with timestamp or count as score)
        \item Regular hash for username-to-ID mapping
    \end{itemize}
    
    \item \textbf{Indexing for fast lookups}: Secondary indices are created for frequently queried attributes:
    \begin{itemize}
        \item Username-to-ID mapping for quick user lookups
        \item Temporary sorted sets for engagement-based rankings
    \end{itemize}
    
    \item \textbf{Score-based sorting}: Timestamps and counts are used as scores in sorted sets for efficient ranking:
    \begin{itemize}
        \item Timestamps for chronological timeline sorting
        \item Follower counts for popularity ranking
        \item Engagement metrics for trending content
    \end{itemize}
\end{enumerate}

\subsection{Data Models}

\subsubsection{User Model}
Users are modeled using Redis hashes with the key pattern \texttt{users:\{user\_id\}}:

\begin{lstlisting}[language=Python, caption=User data structure in Redis]
# User hash example (users:123456789)
{
    "username": "testuser",
    "name": "Test User",
    "follower_count": 100,
    "following_count": 50,
    "chirp_count": 200,
    "created_at": "Mon Apr 01 12:00:00 +0000 2025",
    "profile_image": "https://example.com/image.jpg"
}
\end{lstlisting}

For quick username lookups, a separate hash maps usernames to user IDs:

\begin{lstlisting}[language=Python, caption=Username to user ID mapping]
# Username index (usernames)
{
    "testuser": "123456789",
    "anotheruser": "987654321",
    ...
}
\end{lstlisting}

\subsubsection{Chirp Model}
Chirps (tweets) are stored as hashes with the key pattern \texttt{chirp:\{chirp\_id\}}:

\begin{lstlisting}[language=Python, caption=Chirp data structure in Redis]
# Chirp hash example (chirp:987654321)
{
    "text": "This is a test chirp!",
    "user_id": "123456789",
    "username": "testuser",
    "created_at": "Mon Apr 01 12:30:00 +0000 2025",
    "lang": "en",
    "favorite_count": 10,
    "retweet_count": 5
}
\end{lstlisting}

\subsubsection{Timeline and Rankings}
Redis sorted sets are used for the timeline and user rankings:

\begin{lstlisting}[language=Python, caption=Timeline and ranking data structures]
# Global timeline (chirps:timeline)
# Format: chirp_id -> timestamp
# Sorted by timestamp for chronological order
{
    "987654321": 1712055000.0,
    "987654322": 1712055060.0,
    ...
}

# Top users by followers (users:top_followers)
# Format: user_id -> follower_count
# Sorted by follower count
{
    "123456789": 100,
    "987654321": 200,
    ...
}

# Top users by chirp count (users:top_posters)
# Format: user_id -> chirp_count
# Sorted by chirp count
{
    "123456789": 200,
    "987654321": 150,
    ...
}
\end{lstlisting}

\subsection{Data Relationships}
Relationships between entities are modeled through:

\begin{enumerate}
    \item \textbf{Reference by ID}: Chirps contain user\_id to establish ownership
    \item \textbf{Denormalized fields}: Usernames are duplicated in chirp records for performance
    \item \textbf{Counters}: Follower/following counts are maintained in user records
    \item \textbf{Sorted sets}: Used to maintain relationships with additional metadata (timestamps, counts)
\end{enumerate}

This approach differs from traditional relational databases where relationships would be maintained through foreign keys and joins. In our Redis implementation, denormalization plays a critical role in achieving performance at scale. The trade-off is increased storage requirements and the need to ensure data consistency when updating records (for example, when a user posts a new chirp, we must update both the timeline and the user's chirp count).

\newpage
\section{Software Architecture and Functionalities}

\subsection{System Architecture}
The Chirp application is organized into a layered architecture:

\begin{figure}[H]
    \centering
    \begin{tabular}{|c|}
        \hline
        \textbf{Presentation Layer} \\
        Command-line Interface (CLI) / Streamlit Web Interface \\
        \hline
        \textbf{Application Layer} \\
        ChirpApp (Business Logic) \\
        \hline
        \textbf{Data Access Layer} \\
        ChirpRedisModel (Data Model) \\
        \hline
        \textbf{Database} \\
        Redis \\
        \hline
    \end{tabular}
    \caption{Architectural layers of the Chirp application}
\end{figure}

The project is structured as follows:

\begin{verbatim}
chirp-redis-keyvalue-lab/
|-- src/                     # Main source code 
|   |-- app/                 # Application code
|   |   |-- __init__.py
|   |   |-- chirp_app.py     # Command-line application
|   |   `-- streamlit_app.py # Web application
|   `-- models/              # Redis data models
|       |-- __init__.py      
|       `-- redis_model.py   # Core Redis data model implementation
|-- scripts/                 # Utility scripts
|   |-- import_data.py       # Data import script
|   |-- process_jsonl.py     # Data processing script
|   |-- reset_db.py          # Database reset script
|   |-- run_app.py           # Application launcher
|   `-- fix_engagement.py    # Script to add engagement metrics
|-- data/                    # Generated data
|   `-- processed/           # Processed data directory
`-- tests/                   # Test suite
    |-- __init__.py
    |-- conftest.py
    |-- test_redis_model.py
    |-- test_import_data.py
    `-- test_streamlit_app.py
\end{verbatim}

\subsection{Core Components}

\subsubsection{Data Model (ChirpRedisModel)}
The data model class encapsulates all interactions with Redis, providing an abstraction layer for the application logic:

\begin{lstlisting}[language=Python, caption=Key methods in the ChirpRedisModel class]
class ChirpRedisModel:
    # Core methods
    def import_user(self, user_data): ...
    def import_chirp(self, chirp_data): ...
    def get_latest_chirps(self, count=5): ...
    def get_top_users_by_followers(self, count=5): ...
    def get_top_posters(self, count=5): ...
    def post_chirp(self, user_id, text): ...
    def like_chirp(self, chirp_id): ...
    def rechirp(self, chirp_id): ...
    def add_user(self, username, name, profile_image=''): ...
    def get_top_liked_chirps(self, count=5): ...
    def get_top_rechirped_chirps(self, count=5): ...
    def reset_db(self): ...
\end{lstlisting}

\subsubsection{Command-line Interface (ChirpApp)}
The CLI provides an interactive interface to the Chirp functionality:

\begin{lstlisting}[language=Python, caption=ChirpApp class structure]
class ChirpApp:
    def __init__(self, host='localhost', port=6379, db=0): ...
    def display_welcome(self): ...
    def display_help(self): ...
    def format_chirp(self, chirp): ...
    def format_user(self, user): ...
    def display_latest_chirps(self): ...
    def display_top_followers(self): ...
    def display_top_posters(self): ...
    def post_new_chirp(self, username, text): ...
    def like_chirp(self, chirp_id): ...
    def rechirp(self, chirp_id): ...
    def add_new_user(self, username, name): ...
    def display_top_liked(self): ...
    def display_top_rechirped(self): ...
    def run(self): ...
\end{lstlisting}

\subsubsection{Web Interface (Streamlit)}
The Streamlit app provides a user-friendly web interface:

\begin{lstlisting}[language=Python, caption=Streamlit app structure]
# Initialize the Redis model
@st.cache_resource
def get_model(): ...

# Main page components
st.title("Chirp")
st.subheader("Compact Hub for Instant Real-time Posting")

# Navigation
page = st.sidebar.radio("Go to", ["Home", "Post a Chirp", "Top Users", "About"])

# Page implementations (Home, Post a Chirp, Top Users, About)
if page == "Home":
    # Display latest chirps, most liked, most rechirped
    ...
elif page == "Post a Chirp":
    # Form to post new chirps or add users
    ...
elif page == "Top Users":
    # Display top users by followers and chirps
    ...
elif page == "About":
    # About page content
    ...
\end{lstlisting}

\subsection{Data Flow}

\subsubsection{Posting a New Chirp}
The process of posting a new chirp involves:

\begin{enumerate}
    \item The user submits text content via CLI or web interface
    \item The application layer validates the input and calls the data model
    \item The data model:
        \begin{itemize}
            \item Generates a unique chirp ID based on timestamp
            \item Creates a chirp hash with the text, user information, and metadata
            \item Adds the chirp to the timeline sorted set with the timestamp as score
            \item Increments the user's chirp count
            \item Updates the top posters ranking
        \end{itemize}
    \item The application confirms the operation with feedback to the user
\end{enumerate}

\subsubsection{Retrieving Latest Chirps}
When fetching the latest chirps:

\begin{enumerate}
    \item The application requests latest chirps from the data model
    \item The data model:
        \begin{itemize}
            \item Queries the timeline sorted set for the most recent chirp IDs
            \item Retrieves the full chirp data for each ID
            \item Formats and returns the complete chirp objects
        \end{itemize}
    \item The application layer formats the chirps for display
    \item The interface presents the formatted chirps to the user
\end{enumerate}

\subsection{Data Import Pipeline}
A data import pipeline was developed to process Twitter data:

\begin{enumerate}
    \item \textbf{processing.py} - Extracts English tweets from compressed JSON files
    \item \textbf{import\_data.py} - Filters and imports tweets into Redis
    \item \textbf{fix\_engagement.py} - Adds realistic engagement metrics
\end{enumerate}

\subsection{Implemented Functionalities}

\subsubsection{Core Features}
\begin{itemize}
    \item User profile management (creation, statistics)
    \item Posting new chirps (text-only messages)
    \item Viewing latest chirps timeline
    \item Viewing top users by followers
    \item Viewing top users by post count
\end{itemize}

\subsubsection{Additional Features}
\begin{itemize}
    \item Engagement metrics (likes, rechirps)
    \item Top chirps by engagement (most liked, most rechirped)
    \item Web interface with Streamlit
    \item Data import capabilities
    \item Database reset functionality
\end{itemize}

\newpage
\section{Testing}

\subsection{Testing Strategy}
A focused testing approach was implemented with:

\begin{itemize}
    \item Unit tests for the Redis model core functionality
    \item Integration tests for the data import process
    \item Web interface tests using mocks for the model interactions
\end{itemize}

The testing framework uses:

\begin{itemize}
    \item pytest as the test runner
    \item fakeredis for Redis mocking to avoid actual database operations
    \item unittest.mock for additional component mocking
    \item pytest-cov for coverage reporting and monitoring
\end{itemize}

\subsection{Test Structure}
The test suite is organized into three main modules:

\begin{lstlisting}[language=bash, caption=Test suite organization]
tests/
|-- test_import_data.py     # Tests for the data import process
|-- test_redis_model.py     # Tests for the Redis data model
`-- test_streamlit_app.py   # Tests for the web interface
\end{lstlisting}

\subsection{Test Coverage}
The current test suite includes 18 tests that focus on key functionality:

\begin{itemize}
    \item \textbf{Redis model (10 tests)}: Core data model operations including user and chirp management, timeline generation, and engagement functionality
    \item \textbf{Import process (4 tests)}: Data import validation, filtering, and error handling
    \item \textbf{Web interface (4 tests)}: Key user interactions with the Streamlit application
\end{itemize}

\subsection{Testing Results}
The test suite covers the core functionality with a focus on the data layer. Results from the most recent test run:

\begin{lstlisting}[language=bash, caption=Test execution results]
============================================== test session starts ===============================================
platform linux -- Python 3.10.12, pytest-7.4.0, pluggy-1.5.0
plugins: cov-4.1.0, mock-3.11.1
collected 18 items                                                                                              

tests/test_import_data.py::TestDataImport::test_import_english_tweets_only PASSED                         [  5%]
tests/test_import_data.py::TestDataImport::test_import_with_limit PASSED                                  [ 11%]
tests/test_import_data.py::TestDataImport::test_import_with_engagement PASSED                             [ 16%]
tests/test_import_data.py::TestDataImport::test_import_handles_json_decode_error PASSED                   [ 22%]
tests/test_redis_model.py::TestChirpRedisModel::test_import_user PASSED                                   [ 27%]
tests/test_redis_model.py::TestChirpRedisModel::test_import_chirp PASSED                                  [ 33%]
tests/test_redis_model.py::TestChirpRedisModel::test_get_latest_chirps PASSED                             [ 38%]
tests/test_redis_model.py::TestChirpRedisModel::test_get_top_users_by_followers PASSED                    [ 44%]
tests/test_redis_model.py::TestChirpRedisModel::test_post_chirp PASSED                                    [ 50%]
tests/test_redis_model.py::TestChirpRedisModel::test_like_chirp PASSED                                    [ 55%]
tests/test_redis_model.py::TestChirpRedisModel::test_rechirp PASSED                                       [ 61%]
tests/test_redis_model.py::TestChirpRedisModel::test_add_user PASSED                                      [ 66%]
tests/test_redis_model.py::TestChirpRedisModel::test_get_top_liked_chirps PASSED                          [ 72%]
tests/test_redis_model.py::TestChirpRedisModel::test_import_english_tweets_only PASSED                    [ 77%]
tests/test_streamlit_app.py::TestStreamlitApp::test_model_get_latest_chirps PASSED                        [ 83%]
tests/test_streamlit_app.py::TestStreamlitApp::test_model_post_chirp PASSED                               [ 88%]
tests/test_streamlit_app.py::TestStreamlitApp::test_model_like_chirp PASSED                               [ 94%]
tests/test_streamlit_app.py::TestStreamlitApp::test_model_add_user PASSED                                 [100%]
\end{lstlisting}

The key coverage metrics show a strategic focus on the data model:

\begin{itemize}
    \item Total tests: 18
    \item Total execution time: 0.62 seconds
    \item Core data model coverage: 75\%
    \item Data import script coverage: 62\%
    \item Overall project coverage: 23\%
\end{itemize}

This coverage pattern reflects a deliberate prioritization of testing the data layer, which contains the most business-critical logic. Future test development would focus on increasing coverage of the UI components and utility scripts.

\newpage
\section{Conclusion and Future Work}

\subsection{Achievements}
This project successfully implemented:

\begin{itemize}
    \item A comprehensive key-value data model for a social media application
    \item Efficient data structures for timeline, user management, and rankings
    \item Both command-line and web interfaces
    \item Data import functionality
    \item Additional engagement features
\end{itemize}

The implementation demonstrates:

\begin{itemize}
    \item Effective use of Redis data structures (hashes, sorted sets)
    \item Appropriate denormalization strategies
    \item Performance optimization through careful key design
    \item Clean separation of concerns in architecture
\end{itemize}

\subsection{Challenges and Limitations}
Several challenges were encountered:

\begin{itemize}
    \item Modeling relationships in a key-value store requires denormalization
    \item Managing sorted sets for rankings requires careful transaction management
    \item Limited options for complex queries compared to relational databases
    \item Memory consumption of denormalized data
\end{itemize}

Current limitations of the implementation:

\begin{itemize}
    \item No support for media content (images, videos)
    \item Limited user authentication and security
    \item No support for hashtags or mentions
    \item No support for chirp deletion or editing
    \item Limited search capabilities
\end{itemize}

\subsection{Future Work}
Potential improvements for future iterations include both feature expansions and technical improvements:

\begin{itemize}
    \item \textbf{Feature enhancements:}
    \begin{itemize}
        \item Implementing a follow/unfollow mechanism
        \item Adding personalized user timelines
        \item Supporting hashtags and topic trending
        \item Implementing user mentions and notifications
        \item Adding search functionality
        \item Supporting chirp deletion and editing
        \item Adding media content support
    \end{itemize}
    
    \item \textbf{Technical improvements:}
    \begin{itemize}
        \item Improving test coverage for UI components and utility scripts
        \item Implementing continuous integration via GitHub Actions
        \item Improving security with authentication
        \item Implementing data sharding for scalability
        \item Adding automated performance benchmarking
        \item Implementing data backup and recovery mechanisms
    \end{itemize}
\end{itemize}

\subsection{Performance Considerations}
A key aspect of future work would involve benchmarking and optimizing performance:

\begin{itemize}
    \item Load testing with variable user counts and chirp volumes
    \item Memory usage optimization techniques
    \item Implementing caching strategies for frequently accessed data
    \item Exploring Redis Cluster for horizontal scaling
\end{itemize}

\subsection{Lessons Learned}
This project provided valuable insights into:

\begin{itemize}
    \item Non-relational data modeling approaches
    \item Performance implications of different Redis data structures
    \item Trade-offs between normalization and query performance
    \item Importance of key design in key-value databases
    \item Benefits and limitations of key-value stores for social media applications
\end{itemize}

\subsection{Conclusion}
The Chirp project demonstrates that key-value databases like Redis can effectively support core social media functionalities with excellent performance characteristics. The implementation successfully met all the laboratory requirements while providing additional features to enhance the application's realism and usability.

The denormalized data model and careful selection of Redis data structures enabled efficient operations for timeline retrieval, user rankings, and chirp management. This approach highlights the strengths of key-value databases in handling high-throughput, real-time social media scenarios, while also illustrating the design considerations necessary when moving beyond traditional relational database models.

This project provides a solid foundation for further exploration of non-relational database technologies and their application in building scalable, real-time social platforms.

\end{document}